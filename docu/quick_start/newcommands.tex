%%%%%%%%%%%%%%%%%%%%%%%%%%%%%%%%%%%%%%%%%%%%%%%%%%%%%%%%%%%%%%%%%%%%%%%
%
% This is newcommands.tex by Ralf Greve, 2006-07-27
%
%%%%%%%%%%%%%%%%%%%%%%%%%%%%%%%%%%%%%%%%%%%%%%%%%%%%%%%%%%%%%%%%%%%%%%%

% Mathematical equations:
\newcommand{\beq}{\begin{equation}}
\newcommand{\eeq}{\end{equation}}
\newcommand{\beqa}{\begin{eqnarray}}
\newcommand{\eeqa}{\end{eqnarray}}
\newcommand{\beqan}{\begin{eqnarray*}}
\newcommand{\eeqan}{\end{eqnarray*}}

% Columns and matrices:
\newcommand{\veczd}[2]{\left(\begin{array}{c}{#1}\\{#2}\end{array}\right)}
\newcommand{\vecdd}[3]{\left(\begin{array}{c}{#1}\\{#2}\\{#3}\end{array}\right)}
\newcommand{\vecvd}[4]{\left(\begin{array}{c}{#1}\\{#2}\\{#3}\\{#4}\end{array}\right)}
\newcommand{\matzd}[4]{\left(\begin{array}{cc}{#1}&{#2}\\{#3}&{#4}\end{array}\right)}
\newcommand{\matdd}[9]{\left(\begin{array}{ccc}{#1}&{#2}&{#3}\\%
                                               {#4}&{#5}&{#6}\\%
                                               {#7}&{#8}&{#9}%
                                               \end{array}\right)}

% Displaystyle fractions and big integrals in running text:
\newcommand{\bigfrac}[2]{\mbox{$\displaystyle\frac{#1}{#2}$}}
\newcommand{\bigint}{\displaystyle\int}
\newcommand{\bigoint}{\displaystyle\oint}
\newcommand{\bigsum}{\displaystyle\sum}

% Mathematical abbreviations:
\newcommand{\err}{I\hspace*{-0.375em}R}
% \renewcommand{\d}{\partial}
\newcommand{\pa}{\partial}
\newcommand{\D}{\mathrm{d}}
\newcommand{\E}{\mathrm{e}}
\newcommand{\I}{\mathrm{i}}
\newcommand{\hdot}{^{\displaystyle\cdot}}
\newcommand{\grad}{\mbox{grad}\,}
\newcommand{\divg}{\mbox{div}\,}
\newcommand{\rot}{\mbox{rot}\,}
\newcommand{\curl}{\mbox{curl}\,}
\newcommand{\laplace}{\mathrm{\Delta}}
\newcommand{\Grad}{\mbox{Grad}\,}
\newcommand{\Div}{\mbox{Div}\,}
\newcommand{\Rot}{\mbox{Rot}\,}
\newcommand{\Curl}{\mbox{Curl}\,}
\newcommand{\tr}{\mbox{tr}\,}
% \renewcommand{\Re}{\mbox{Re}\,}
% \renewcommand{\Im}{\mbox{Im}\,}
\newcommand{\Real}{\mbox{Re}\,}
\newcommand{\Imag}{\mbox{Im}\,}
\newcommand{\sym}{\mbox{sym}\,}
\newcommand{\skw}{\mbox{skw}\,}
\newcommand{\eps}{\varepsilon}
\newcommand{\sgs}{\stackrel{!}{=}}  % sgs: "soll gleich sein"
\newcommand{\tra}{^{\mathrm{T}}}    % Transposition
\newcommand{\dev}{^{\mathrm{D}}}    % Deviator
\newcommand{\erf}{\mbox{erf}}
\newcommand{\erfc}{\mbox{erfc}}

% Numerical abbreviations:
\newcommand{\dx}{\Delta x}
\newcommand{\dy}{\Delta y}
\newcommand{\dz}{\Delta z}
\newcommand{\dzc}{\Delta\zeta_c}
\newcommand{\dzt}{\Delta\zeta_t}
\newcommand{\dzr}{\Delta\zeta_r}
\newcommand{\dt}{\Delta t}
\newcommand{\dtt}{\widetilde{\Delta t}}
\newcommand{\tabl}[2]{\frac{\D #1}{\D #2}}
\newcommand{\ttabl}[2]{\frac{\D^2 #1}{\D #2 ^2}}
\newcommand{\pabl}[2]{\frac{\partial #1}{\partial #2}}
\newcommand{\ppabl}[2]{\frac{\partial^2 #1}{\partial #2 ^2}}
\newcommand{\bigtabl}[2]{\bigfrac{\D #1}{\D #2}}
\newcommand{\bigttabl}[2]{\bigfrac{\D^2 #1}{\D #2 ^2}}
\newcommand{\bigpabl}[2]{\bigfrac{\partial #1}{\partial #2}}
\newcommand{\bigppabl}[2]{\bigfrac{\partial^2 #1}{\partial #2 ^2}}

% Further abbreviations:
\newcommand{\absatz}{\vspace*{1ex}}
\newcommand{\nl}{\nonumber\\}
\newcommand{\hf}{\mbox{$\frac{1}{2}$}}               % small 1/2
\newcommand{\onethird}{\mbox{$\frac{1}{3}$}}
\newcommand{\onesixth}{\mbox{$\frac{1}{6}$}}
\newcommand{\twothirds}{\mbox{$\frac{2}{3}$}}
\newcommand{\onenineth}{\mbox{$\frac{1}{9}$}}
\newcommand{\fournineths}{\mbox{$\frac{4}{9}$}}
\newcommand{\degC}{\ensuremath{^\circ\mathrm{C}}}    % degrees C
\newcommand{\degN}{\ensuremath{^\circ\mathrm{N}}}    % degrees N (lat)
\newcommand{\degS}{\ensuremath{^\circ\mathrm{S}}}    % degrees S (lat)
\newcommand{\degE}{\ensuremath{^\circ\mathrm{E}}}    % degrees E (lon)
\newcommand{\degW}{\ensuremath{^\circ\mathrm{W}}}    % degrees W (lon)
\newcommand{\qgeo}{\ensuremath{Q_\mathrm{geoth}^\perp}}
\newcommand{\qg}{\ensuremath{q_\mathrm{geo}}}
\newcommand{\qgp}{\ensuremath{\qg^\perp}}
\newcommand{\qoc}{\ensuremath{Q_\mathrm{oc}^\perp}}
\newcommand{\pdd}{\ensuremath{P\hspace*{-0.083em}D\hspace*{-0.083em}D}}
\newcommand{\Fr}{\ensuremath{F\hspace*{-0.083em}r}}      % Froude number
\newcommand{\Ro}{\ensuremath{R\hspace*{-0.083em}o}}      % Rossby number
\newcommand{\delo}{\ensuremath{\delta^{18}\mathrm{O}}}   % delta-18-O
\newcommand{\deld}{\ensuremath{\delta\mathrm{D}}}        % delta-D

% Jump brackets:
\newcommand{\jl}{\ensuremath{\left[\!\left[}}
\newcommand{\jr}{\ensuremath{\right]\!\right]}}
\newcommand{\jlf}{\ensuremath{[\![}}
\newcommand{\jrf}{\ensuremath{]\!]}}

% Standard definition "poor man's bold":
\def\pmb#1{\setbox0=\hbox{#1}%
  \kern-.025em\copy0\kern-\wd0
  \kern.05em\copy0\kern-\wd0
  \kern-.025em\raise.0433em\box0}

% "poor man's bold" in math mode:
\def\pmbm#1{\setbox0=\hbox{$#1$}%
  \kern-.025em\copy0\kern-\wd0
  \kern.05em\copy0\kern-\wd0
  \kern-.025em\raise.0433em\box0}

% Vectors and tensors in formulas:

% \newcommand{\mbf}[1]{\mbox{\boldmath $#1$}} (old definition of \mbf)

\def\vecbi#1{\relax\ifmmode\mathchoice
  {\mbox{\boldmath$\relax\displaystyle#1$}}
  {\mbox{\boldmath$\relax\textstyle#1$}}
  {\mbox{\boldmath$\relax\scriptstyle#1$}}
  {\mbox{\boldmath$\relax\scriptscriptstyle#1$}}\else
  \hbox{\boldmath$\relax\textstyle#1$}\fi} % boldface italic symbols

\def\vecbu#1{\relax\ifmmode\mathchoice
  {\mbox{\boldmath$\bf\displaystyle#1$}}
  {\mbox{\boldmath$\bf\textstyle#1$}}
  {\mbox{\boldmath$\bf\scriptstyle#1$}}
  {\mbox{\boldmath$\bf\scriptscriptstyle#1$}}\else
  \hbox{\boldmath$\bf\textstyle#1$}\fi}    % boldface upright symbols

\def\tenssu#1{\relax\ifmmode\mathchoice
    {\mbox{$\sf\displaystyle#1$}}
    {\mbox{$\sf\textstyle#1$}}
    {\mbox{$\sf\scriptstyle#1$}}
    {\mbox{$\sf\scriptscriptstyle#1$}}\else
    \hbox{$\sf\textstyle#1$}\fi}           % sans-serif upright symbols
