\documentclass[12pt,a4paper]{article}

\usepackage[DIV=13,BCOR=0mm]{typearea}
\usepackage{setspace}
\usepackage{parskip}

\usepackage{natbib}
\bibpunct[, ]{(}{)}{,}{a}{}{,}

% \usepackage{wasysym}

% Check if we are compiling under latex or pdflatex
% \ifx\pdftexversion\undefined
%  \usepackage[dvips]{graphicx}
% \else
   \usepackage[pdftex]{graphicx}
% \fi

% \graphicspath{{../Figs/}}

\usepackage{hyperref}
\usepackage{hyperxmp}
\hypersetup{
    pdfauthor={Ralf Greve},
    pdftitle={SICOPOLIS V5-dev - Quick Start Manual},
    pdfsubject={Ice sheet model},
    pdfkeywords={Ice sheet, Model, Modelling, %
                 Open source, Free software, SICOPOLIS},
    pdfcopyright={GNU General Public License version 3 or later},
    pdflicenseurl={http://www.gnu.org/licenses/}}

% Positioning of figures
% (values chosen such that figure is inserted at the place of the call
% if possible):
\setcounter{topnumber}{10}
\setcounter{bottomnumber}{10}
\setcounter{totalnumber}{20}
\renewcommand{\topfraction}{0.99}
\renewcommand{\textfraction}{0.01}
\renewcommand{\bottomfraction}{0.99}

\onehalfspacing
% \setstretch{1.22}

%%%%%%%%%%%%%%%%%%%%%%%%%%%%%%%%%%%%%%%%%%%%%%%%%%%%%%%%%%%%%%%%%%%%%%%
%
% This is newcommands.tex by Ralf Greve, 2006-07-27
%
%%%%%%%%%%%%%%%%%%%%%%%%%%%%%%%%%%%%%%%%%%%%%%%%%%%%%%%%%%%%%%%%%%%%%%%

% Mathematical equations:
\newcommand{\beq}{\begin{equation}}
\newcommand{\eeq}{\end{equation}}
\newcommand{\beqa}{\begin{eqnarray}}
\newcommand{\eeqa}{\end{eqnarray}}
\newcommand{\beqan}{\begin{eqnarray*}}
\newcommand{\eeqan}{\end{eqnarray*}}

% Columns and matrices:
\newcommand{\veczd}[2]{\left(\begin{array}{c}{#1}\\{#2}\end{array}\right)}
\newcommand{\vecdd}[3]{\left(\begin{array}{c}{#1}\\{#2}\\{#3}\end{array}\right)}
\newcommand{\vecvd}[4]{\left(\begin{array}{c}{#1}\\{#2}\\{#3}\\{#4}\end{array}\right)}
\newcommand{\matzd}[4]{\left(\begin{array}{cc}{#1}&{#2}\\{#3}&{#4}\end{array}\right)}
\newcommand{\matdd}[9]{\left(\begin{array}{ccc}{#1}&{#2}&{#3}\\%
                                               {#4}&{#5}&{#6}\\%
                                               {#7}&{#8}&{#9}%
                                               \end{array}\right)}

% Displaystyle fractions and big integrals in running text:
\newcommand{\bigfrac}[2]{\mbox{$\displaystyle\frac{#1}{#2}$}}
\newcommand{\bigint}{\displaystyle\int}
\newcommand{\bigoint}{\displaystyle\oint}
\newcommand{\bigsum}{\displaystyle\sum}

% Mathematical abbreviations:
\newcommand{\err}{I\hspace*{-0.375em}R}
% \renewcommand{\d}{\partial}
\newcommand{\pa}{\partial}
\newcommand{\D}{\mathrm{d}}
\newcommand{\E}{\mathrm{e}}
\newcommand{\I}{\mathrm{i}}
\newcommand{\hdot}{^{\displaystyle\cdot}}
\newcommand{\grad}{\mbox{grad}\,}
\newcommand{\divg}{\mbox{div}\,}
\newcommand{\rot}{\mbox{rot}\,}
\newcommand{\curl}{\mbox{curl}\,}
\newcommand{\laplace}{\mathrm{\Delta}}
\newcommand{\Grad}{\mbox{Grad}\,}
\newcommand{\Div}{\mbox{Div}\,}
\newcommand{\Rot}{\mbox{Rot}\,}
\newcommand{\Curl}{\mbox{Curl}\,}
\newcommand{\tr}{\mbox{tr}\,}
% \renewcommand{\Re}{\mbox{Re}\,}
% \renewcommand{\Im}{\mbox{Im}\,}
\newcommand{\Real}{\mbox{Re}\,}
\newcommand{\Imag}{\mbox{Im}\,}
\newcommand{\sym}{\mbox{sym}\,}
\newcommand{\skw}{\mbox{skw}\,}
\newcommand{\eps}{\varepsilon}
\newcommand{\sgs}{\stackrel{!}{=}}  % sgs: "soll gleich sein"
\newcommand{\tra}{^{\mathrm{T}}}    % Transposition
\newcommand{\dev}{^{\mathrm{D}}}    % Deviator
\newcommand{\erf}{\mbox{erf}}
\newcommand{\erfc}{\mbox{erfc}}

% Numerical abbreviations:
\newcommand{\dx}{\Delta x}
\newcommand{\dy}{\Delta y}
\newcommand{\dz}{\Delta z}
\newcommand{\dzc}{\Delta\zeta_c}
\newcommand{\dzt}{\Delta\zeta_t}
\newcommand{\dzr}{\Delta\zeta_r}
\newcommand{\dt}{\Delta t}
\newcommand{\dtt}{\widetilde{\Delta t}}
\newcommand{\tabl}[2]{\frac{\D #1}{\D #2}}
\newcommand{\ttabl}[2]{\frac{\D^2 #1}{\D #2 ^2}}
\newcommand{\pabl}[2]{\frac{\partial #1}{\partial #2}}
\newcommand{\ppabl}[2]{\frac{\partial^2 #1}{\partial #2 ^2}}
\newcommand{\bigtabl}[2]{\bigfrac{\D #1}{\D #2}}
\newcommand{\bigttabl}[2]{\bigfrac{\D^2 #1}{\D #2 ^2}}
\newcommand{\bigpabl}[2]{\bigfrac{\partial #1}{\partial #2}}
\newcommand{\bigppabl}[2]{\bigfrac{\partial^2 #1}{\partial #2 ^2}}

% Further abbreviations:
\newcommand{\absatz}{\vspace*{1ex}}
\newcommand{\nl}{\nonumber\\}
\newcommand{\hf}{\mbox{$\frac{1}{2}$}}               % small 1/2
\newcommand{\onethird}{\mbox{$\frac{1}{3}$}}
\newcommand{\onesixth}{\mbox{$\frac{1}{6}$}}
\newcommand{\twothirds}{\mbox{$\frac{2}{3}$}}
\newcommand{\onenineth}{\mbox{$\frac{1}{9}$}}
\newcommand{\fournineths}{\mbox{$\frac{4}{9}$}}
\newcommand{\degC}{\ensuremath{^\circ\mathrm{C}}}    % degrees C
\newcommand{\degN}{\ensuremath{^\circ\mathrm{N}}}    % degrees N (lat)
\newcommand{\degS}{\ensuremath{^\circ\mathrm{S}}}    % degrees S (lat)
\newcommand{\degE}{\ensuremath{^\circ\mathrm{E}}}    % degrees E (lon)
\newcommand{\degW}{\ensuremath{^\circ\mathrm{W}}}    % degrees W (lon)
\newcommand{\qgeo}{\ensuremath{Q_\mathrm{geoth}^\perp}}
\newcommand{\qg}{\ensuremath{q_\mathrm{geo}}}
\newcommand{\qgp}{\ensuremath{\qg^\perp}}
\newcommand{\qoc}{\ensuremath{Q_\mathrm{oc}^\perp}}
\newcommand{\pdd}{\ensuremath{P\hspace*{-0.083em}D\hspace*{-0.083em}D}}
\newcommand{\Fr}{\ensuremath{F\hspace*{-0.083em}r}}      % Froude number
\newcommand{\Ro}{\ensuremath{R\hspace*{-0.083em}o}}      % Rossby number
\newcommand{\delo}{\ensuremath{\delta^{18}\mathrm{O}}}   % delta-18-O
\newcommand{\deld}{\ensuremath{\delta\mathrm{D}}}        % delta-D

% Jump brackets:
\newcommand{\jl}{\ensuremath{\left[\!\left[}}
\newcommand{\jr}{\ensuremath{\right]\!\right]}}
\newcommand{\jlf}{\ensuremath{[\![}}
\newcommand{\jrf}{\ensuremath{]\!]}}

% Standard definition "poor man's bold":
\def\pmb#1{\setbox0=\hbox{#1}%
  \kern-.025em\copy0\kern-\wd0
  \kern.05em\copy0\kern-\wd0
  \kern-.025em\raise.0433em\box0}

% "poor man's bold" in math mode:
\def\pmbm#1{\setbox0=\hbox{$#1$}%
  \kern-.025em\copy0\kern-\wd0
  \kern.05em\copy0\kern-\wd0
  \kern-.025em\raise.0433em\box0}

% Vectors and tensors in formulas:

% \newcommand{\mbf}[1]{\mbox{\boldmath $#1$}} (old definition of \mbf)

\def\vecbi#1{\relax\ifmmode\mathchoice
  {\mbox{\boldmath$\relax\displaystyle#1$}}
  {\mbox{\boldmath$\relax\textstyle#1$}}
  {\mbox{\boldmath$\relax\scriptstyle#1$}}
  {\mbox{\boldmath$\relax\scriptscriptstyle#1$}}\else
  \hbox{\boldmath$\relax\textstyle#1$}\fi} % boldface italic symbols

\def\vecbu#1{\relax\ifmmode\mathchoice
  {\mbox{\boldmath$\bf\displaystyle#1$}}
  {\mbox{\boldmath$\bf\textstyle#1$}}
  {\mbox{\boldmath$\bf\scriptstyle#1$}}
  {\mbox{\boldmath$\bf\scriptscriptstyle#1$}}\else
  \hbox{\boldmath$\bf\textstyle#1$}\fi}    % boldface upright symbols

\def\tenssu#1{\relax\ifmmode\mathchoice
    {\mbox{$\sf\displaystyle#1$}}
    {\mbox{$\sf\textstyle#1$}}
    {\mbox{$\sf\scriptstyle#1$}}
    {\mbox{$\sf\scriptscriptstyle#1$}}\else
    \hbox{$\sf\textstyle#1$}\fi}           % sans-serif upright symbols


%%%%%%%%%%%%%%%%%%%%%%%%%%%%%%%%%%%%%%%%%%%%%%%%%%%%%%%%%%%%%%%%%%%%%%%%

\begin{document}

\begin{center}

\rule{0mm}{15mm}

\begin{huge}
  \textbf{SICOPOLIS V5-dev\\[-0.5ex]
          -- Quick Start Manual --}\\[-0.5ex]
\end{huge}

\rule{0mm}{25mm}

\begin{Large}
  \textsc{Ralf Greve}
\end{Large}

\rule{0mm}{15mm}

\begin{large}
  Institute of Low Temperature Science, Hokkaido University,\\[-0.25ex]
  Kita-19, Nishi-8, Kita-ku, Sapporo 060-0819, Japan\\[1ex]
\end{large}

\vfill

\begin{large}\today\end{large}

\end{center}

\rule{0mm}{20mm}

\clearpage

\rule{0mm}{0mm}

\vfill

\begin{center}\begin{minipage}{0.85\textwidth}

\begin{small}

\noindent{}Copyright 2009--2022 Ralf Greve\\
(with contributions by Jorge Bernales, Sebastian Beyer, Heinz Blatter, Reinhard Calov, Thorben Dunse, Eduardo Flandez, Ben Galton-Fenzi, Thomas G\"olles, Bj\"orn Grieger, Philipp Hancke, Patrick Heimbach, Nina Kirchner, Thomas Kleiner, Sascha Knell, Anne Le Brocq, Liz Curry Logan, Sri Hari Krishna Narayanan, Alex Robinson, Fuyuki Saito, Tatsuru Sato, Marius Sch\"afer, Matthias Scheiter, Oliver J. Stenzel, Malte Thoma, Roland Warner)

\vspace*{1.5ex}

\noindent{}This file is part of SICOPOLIS.

\vspace*{1.5ex}

\noindent{}SICOPOLIS is free software. It can be redistributed and/or modified under the terms of the GNU General Public License (http://www.gnu.org/licenses/) as published by the Free Software Foundation, either version~3 of the License, or (at the user's option) any later version.

\vspace*{1.5ex}

\noindent{}SICOPOLIS is distributed in the hope that it will be useful, but WITHOUT ANY WARRANTY; without even the implied warranty of MERCHANTABILITY or FITNESS FOR A PARTICULAR PURPOSE.  See the GNU General Public License for more details.

\end{small}

\end{minipage}\end{center}

\vfill

\rule{0mm}{0mm}

\clearpage

\section{Requirements}

\begin{itemize}

\item Unix-like system (e.g., Linux).

\item Fortran compiler.
\\
So far, the GNU GCC (gfortran) and Intel Fortran (ifort) compilers are supported. If you wish to use a different compiler, please contact $<$help@sicopolis.net$>$.

\item SICOPOLIS writes output in NetCDF format (plus some ASCII). An installation of NetCDF version 3.6.x or newer (https://www.unidata.ucar.edu/software/netcdf/) is therefore required. For installation support, see Appendix~\ref{sect_install_nc}.

\item For the shallow-shelf/shelfy-stream solver, the Library of Iterative Solvers for Linear Systems (Lis, version 1.4.43 or newer) is required (https://www.ssisc.org/lis/). For installation support, see Appendix~\ref{sect_install_lis}.

\end{itemize}

\section{Download}

\begin{itemize}

\item \textbf{Using Git}

The Git repository of SICOPOLIS is kindly hosted by the GitLab system of the Alfred Wegener Institute for Polar and Marine Research (AWI) in Bremerhaven, Germany. Front page: https://gitlab.awi.de/sicopolis/sicopolis/\,.

Cloning the latest develop revision:

% \begin{small}
\hspace*{1.5em}\verb+git clone --branch develop \+
\\[-1ex]
\hspace*{3.65em}\verb+https://gitlab.awi.de/sicopolis/sicopolis.git+
% \end{small}

(Cloning with SSH instead of HTTPS is also available. See the above GitLab front page link for details.)

You should then have a new directory ``sicopolis'' that contains the entire program package.

\end{itemize}

% \begin{spacing}{1.2}

\section{Initial configuration}

\begin{enumerate}

\item Go to the new directory ``sicopolis'' and execute the following bash scripts:

% \begin{small}
\hspace*{1.5em}\verb+./copy_templates.sh+
\\
\hspace*{1.5em}\verb+./get_input_files.sh+
% \end{small}

The latter can be configured if you want to download only selected input files (default is downloading everything). To do so, open it with a text editor and change the flag variables before execution.

\item Locate the file sico\_configs.sh in the directory sicopolis/runs, and open it with a text editor.

\item Set the flags
\\
LIS\_FLAG, OPENMP\_FLAG and LARGE\_DATA\_\-FLAG 
\\
according to your needs. 

Default is ``true''/``true''/``false'', which works for all test simulations included in the SICOPOLIS package. LIS\_FLAG and OPENMP\_FLAG can be set to ``false'' for simulations with pure shallow-ice dynamics. However, ``true'' is required for simulations with shallow-shelf dynamics (for floating ice) or hybrid shallow-ice--shelfy-stream dynamics (for grounded ice). For high-resolution simulations (e.g., Greenland/5\,km or Antarctica/8\,km), LARGE\_DATA\_FLAG must be set to ``true''.

\item Set NETCDFHOME to the correct path of your NetCDF installation.

If LIS\_FLAG\,=\,``true'', set LISHOME to the correct path of your Lis installation.

\item Depending on your system, some additional settings might have to be added in sico\_configs.sh (\verb+module load+\, commands for dynamic loading etc.).

\item Locate the file sico\_environment.sh in the directory sicopolis/runs, open it with a text editor, and replace the ``Default'' entry for SICO\_INSTITUTION by the name of your institution (max.\ 256 characters).

\end{enumerate}

% \end{spacing}

% \clearpage

% \begin{spacing}{1.2}

\section{Files and directories in ``sicopolis''}

\begin{itemize}

\item \textbf{runs}:

Configuration file sico\_configs.sh.

Shell script (bash) sico.sh for running a single simulation.

Shell scripts (bash) multi\_sico\_1.sh and multi\_sico\_2.sh for running multiple simulations by repeated calls of sico.sh.

Subdirectory \textbf{headers}: specification files sico\_specs\_\emph{run\_name}.h (\emph{run\_name}: name of run) for a number of computationally rather inexpensive test runs.

\begin{itemize}

\item
Run v5\_vialov3d25
\\
$\longrightarrow$ 3-d version of the 2-d ``Vialov profile'' \citep{vialov_1958},
\\
\phantom{$\longrightarrow$} resolution 25~km, $t=0\ldots{}100\,\mathrm{ka}$.
\\
\phantom{$\longrightarrow$} Similar to the EISMINT Phase~1 fixed-margin experiment
\\
\phantom{$\longrightarrow$} \citep{huybrechts_etal_1996}, but without thermodynamics. Instead,
\\
\phantom{$\longrightarrow$} isothermal conditions with $T=-10\degC$ everywhere are assumed.

\item
Run v5\_emtp2sge25\_expA
\\
$\longrightarrow$ EISMINT Phase~2 Simplified Geometry Experiment A,
\\
\phantom{$\longrightarrow$} resolution 25~km, $t=0\ldots{}200\,\mathrm{ka}$ \citep{payne_etal_2000}.
\\
\phantom{$\longrightarrow$} The thermodynamics solver for this run is the one-layer melting-CTS
\\
\phantom{$\longrightarrow$} enthalpy scheme (ENTM), while all other runs employ the 
\\
\phantom{$\longrightarrow$} polythermal two-layer scheme (POLY) \citep{greve_blatter_2016}.

\item
Run v5\_grl20\_ss25ka
\\
$\longrightarrow$ Greenland ice sheet, resolution 20~km,
\\
\phantom{$\longrightarrow$} short steady-state run ($t=0\ldots{}25\,\mathrm{ka}$) for modern climate conditions
\\
\phantom{$\longrightarrow$} (unpublished).

\item
Run v5\_ant40\_ss25ka
\\
$\longrightarrow$ Antarctic ice sheet without ice shelves, resolution 40~km,
\\
\phantom{$\longrightarrow$} short steady-state run ($t=0\ldots{}25\,\mathrm{ka}$) for modern climate conditions
\\
\phantom{$\longrightarrow$} (unpublished).

\item
Run v5\_grl20\_b2\_paleo21
\\
$\longrightarrow$ Greenland ice sheet, resolution 20~km,
\\
\phantom{$\longrightarrow$} $t=-140\ldots{}0\,\mathrm{ka}$, basal sliding ramped up during the first 5\,ka
\\
\phantom{$\longrightarrow$} [modified, low-resolution version of the spin-up for ISMIP6 InitMIP; 
\\
\phantom{$\longrightarrow$} \citet{greve_etal_2017a}].

\item
Runs v5\_grl10\_b2\_paleo21 and
\\{}
\phantom{Runs} v5\_grl10\_b2\_future21\_ctrl/...\_asmb
\\
$\longrightarrow$ Greenland ice sheet, resolution 10~km,
\\
\phantom{$\longrightarrow$} $t=-9\ldots{}0\,\mathrm{ka}$ for the paleo run, $t=0\ldots{}100\,\mathrm{a}$ for the future runs
\\{}
\phantom{$\longrightarrow$} [10-km version of the spin-up and the schematic future climate runs for 
\\
\phantom{$\longrightarrow$} ISMIP6 InitMIP; \citet{greve_etal_2017a}].

\item
Runs v5\_ant64\_b2\_spinup09\_init100a,
\\
\phantom{Runs} v5\_ant64\_b2\_spinup09\_fixtopo, v5\_ant64\_b2\_spinup09 and
\\
\phantom{Runs} v5\_ant64\_b2\_future09\_ctrl
\\
$\longrightarrow$ Antarctic ice sheet with hybrid shallow-ice--shelfy-stream dynamics
\\
\phantom{$\longrightarrow$} \citep{bernales_etal_2017} and ice shelves, resolution 64~km,
\\
\phantom{$\longrightarrow$} $t=-140.1\ldots{}-140.0\,\mathrm{ka}$ for the init run without basal sliding (...\_init100a),
\\
\phantom{$\longrightarrow$} $t=-140\ldots{}0\,\mathrm{ka}$ for the run with almost fixed topography (...\_fixtopo),
\\
\hspace*{9.15em} basal sliding ramped up during the first 5\,ka,
\\
\phantom{$\longrightarrow$} $t=-0.5\ldots{}0\,\mathrm{ka}$ for the final, freely-evolving-topography part of the
\\
\hspace*{9.0em} spin-up (...\_spinup09),
\\
\phantom{$\longrightarrow$} $t=0\ldots{}100\,\mathrm{a}$ for the constant-climate control run (...\_future09\_ctrl)
\\{}
\phantom{$\longrightarrow$} [64-km version of the spin-up and the constant-climate control run for 
\\
\phantom{$\longrightarrow$} ISMIP6 InitMIP; Greve and Galton-Fenzi (pers.\ comm.\ 2017)].

\item
Runs v5\_asf2\_steady and v5\_asf2\_surge
\\
$\longrightarrow$ Austfonna, resolution 2~km, $t=0\ldots{}10\,\mathrm{ka}$
\\
\phantom{$\longrightarrow$} [similar to \citeauthor{dunse_etal_2011}'s (\citeyear{dunse_etal_2011}) Exp.~2 (steady fast flow) and
\\
\phantom{$\longrightarrow$} Exp.~5 (surging-type flow), respectively].

\item
Runs v5\_nmars10\_steady and v5\_smars10\_steady
\\
$\longrightarrow$ North-/south-polar cap of Mars, resolution 10~km, $t=-10\,\mathrm{Ma}\ldots{}0$
\\
\phantom{$\longrightarrow$} [steady-state runs by \citet{greve_2007b}].

\item
Run v5\_nhem80\_nt012\_new
\\
$\longrightarrow$ northern hemisphere, resolution 80~km, $t=-250\ldots{}0\,\mathrm{ka}$
\\
\phantom{$\longrightarrow$} [similar to run nt012 by \citet{greve_etal_1999a}].

\item
Run v5\_heino50\_st
\\
$\longrightarrow$ ISMIP HEINO standard run ST,
\\
\phantom{$\longrightarrow$} resolution 50~km, $t=0\ldots{}200\,\mathrm{ka}$ \citep{calov_etal_2010}.

\end{itemize}

\item \textbf{src}:

Directory that contains the main program file sicopolis.F90.

\begin{itemize}

\item
Subdirectory \textbf{subroutines/general}: general subroutines, for any modelled domain.

\item
Subdirectory \textbf{subroutines/ant}: subroutines specific for the Antarctic ice sheet.

\item
Subdirectory \textbf{subroutines/emtp2sge}: subroutines specific for the EISMINT Phase~2 Simplified Geometry Experiments.

\item
Subdirectory \textbf{subroutines/grl}: subroutines specific for the Greenland ice sheet.

\item
Accordingly subdirectories subroutines/asf, nhem, scand, tibet, nmars and smars for Austfonna, the northern hemisphere, Scandinavia, Tibet and the north and south polar caps of Mars, respectively.

\item
Subdirectory \textbf{subroutines/xyz}: see Appendix~\ref{sect_xyz}.

\end{itemize}

\item \textbf{sico\_in}:

Directory that contains input data files for SICOPOLIS.

\begin{itemize}

\item
Subdirectory \textbf{general}: general input files, for any modelled domain.

\item
Subdirectory \textbf{ant}: input files specific for the Antarctic ice sheet.

\item
Subdirectory \textbf{emtp2sge}: input files specific for the EISMINT Phase~2 Simplified Geometry Experiments.

\item
Subdirectory \textbf{grl}: input files specific for the Greenland ice sheet.

\item
Accordingly subdirectories asf, nhem, scand, tibet, nmars and smars for Austfonna, the northern hemisphere, Scandinavia, Tibet and the north and south polar caps of Mars, respectively.

\item
Subdirectory \textbf{xyz}: see Appendix~\ref{sect_xyz}.

\end{itemize}

\item \textbf{sico\_out}:

Empty directory into which output files of SICOPOLIS simulations are written.

\item \textbf{docu}:

Directory that contains some documentation.

\begin{itemize}

\item
Subdirectory \textbf{quick\_start}:\\
\LaTeX{} source for this manual (PDF must be built with make).

\item
Subdirectory \textbf{doxygen}: documentation to be created by Doxygen\\
(optional, see doxygen-config/README.txt).

\begin{itemize}

\item html/index.html $\longrightarrow$ Source code browser.

\item latex/refman.pdf $\longrightarrow$ Reference manual.

\end{itemize}

\end{itemize}

\item \textbf{tools}:

See Sect.~\ref{sect_tools}.

\item \textbf{license}:

Directory that contains a copy of the GNU General Public License (version~3).

\end{itemize}

% \end{spacing}

\section{How to run a simulation}

\begin{enumerate}

\item For example, to run simulation v5\_grl20\_ss25ka, use the script sico.sh:
\\
\hspace*{10mm}\verb+(./sico.sh -m v5_grl20_ss25ka) >out_001.dat 2>&1 &+
\\
(from directory sicopolis/runs, bash required). Accordingly for the other simulations.
\\
For further options, try \verb+./sico.sh -h+.
\\
\warning{} Do not use out\_\emph{run\_name}.dat for the redirected output of sico.sh. 
\\
\phantom{\warning{}} This name is reserved for the runtime output of SICOPOLIS itself.
\\
\phantom{\warning{} This} (Both are very useful in case of compilation or runtime errors!)

\item Alternatively, if you prefer to run all simulations consecutively, execute the script multi\_sico\_1.sh:
\\
\hspace*{10mm}\verb+(./multi_sico_1.sh) >out_multi_100.dat 2>&1 &+
\\
For further options, try \verb+./multi_sico_1.sh -h+.

\end{enumerate}

\subsubsection*{Computing times}

The approximate computing times for the simulations are listed in Table~\ref{table_comp_times} (Appendix~\ref{sect_table}).

\section{Output files}
\label{sect_output}

Output files are written by default to the directory sicopolis/sico\_out/\emph{run\_name} (this can be changed with the \verb+-d+ option). Four types are produced:

\begin{itemize}

\item \textbf{run\_name.log}:

ASCII file that lists the main specifications of simulation \emph{run\_name}.

\item \textbf{run\_name.ser, run\_name\_ser.nc}:

Time-series files (ASCII, NetCDF) that contain scalar variables:
\begin{itemize}
\item Time, t
\item Surface temperature anomaly, D\_Ts, or glacial index, glac\_ind (forcing)
\item Sea level, z\_sl (forcing)
\item Total ice volume, V
\item Volume of grounded ice, V\_g
\item Volume of floating ice, V\_f
\item Total ice area, A
\item Area of grounded ice, A\_g
\item Area of floating ice, A\_f
\item Ice volume above flotation in sea level equivalent, V\_sle
\item Volume of temperate ice, V\_t
\item Area of temperate-based grounded ice, A\_t
\item Maximum ice thickness, H\_max
\item Maximum thickness of temperate ice, H\_t\_max
\item Maximum surface elevation, zs\_max
\item Maximum surface speed, vs\_max
\item Maximum basal temperature (relative to pmp), Tbh\_max
\item (Some more in the NetCDF file, try \verb+ncdump -h run_name_ser.nc+)
\end{itemize}

\item \textbf{run\_name.core, run\_name\_core.nc}:

Time-series files (ASCII, NetCDF) that contain for selected locations xxx:
\begin{itemize}
\item Time, t
\item Surface temperature anomaly, D\_Ts, or glacial index, glac\_ind (forcing)
\item Sea level, z\_sl (forcing)
\item Thickness, H\_xxx
\item Surface velocity, v\_xxx
\item Basal temperature, T\_xxx
\item (Some more in the NetCDF file, try \verb+ncdump -h run_name_core.nc+)
\end{itemize}

For the Greenland ice sheet, these data are written for seven locations:
\\
GRIP (xxx=GR), GISP2 (xxx=G2), Dye 3 (xxx=D3), Camp Century (xxx=CC), NorthGRIP (xxx=NG), NEEM (xxx=NE), EastGRIP (xxx=EG).

For the Antarctic ice sheet, these data are written for six locations:
\\
Vostok (xxx=Vo), Dome A (xxx=DA), Dome C (xxx=DC), Dome F (xxx=DF), Kohnen (xxx=Ko), Byrd (xxx=By).

\item \textbf{run\_name0001.nc}, \textbf{run\_name0002.nc}, ...:

Complete set of fields (topography, velocity, temperature etc., written in NetCDF (*.nc) format) for selected time slices defined in specifications file.

For example, simulation v5\_grl20\_ss25ka produces three files
\\
v5\_grl20\_ss25ka0001.nc, v5\_grl20\_ss25ka0002.nc and v5\_grl20\_ss25ka0003.nc, which correspond to $t=0$, 10\,ka and 25\,ka, respectively.

\end{itemize}

% \clearpage

% \vfill

\section{Plotting}
\label{sect_plotting}

The output described in Sect.~\ref{sect_output} can be visualized with any plotting tool at the user's preference. Ncview (http://meteora.ucsd.edu/$\sim${}pierce/ncview\_home\_page.html) is a very nice browser for NetCDF files to get a quick and easy look. For more sophisticated plots, one possibility is to use MATLAB, which has an extensive library for NetCDF files (https://www.mathworks.com/help/matlab/network-common-data-form.html). For instance, the following script plots the final surface topography of the Greenland simulation v5\_grl20\_ss25ka (credit: Mathieu Morlighem, University of California Irvine).

\vspace*{2ex}

\noindent\hspace*{5mm}\verb+filename = 'v5_grl20_ss25ka0003.nc';+
\\[-0.5ex]
\hspace*{5mm}\verb+x = ncread(filename,'x');+
\\[-0.5ex]
\hspace*{5mm}\verb+y = ncread(filename,'y');+
\\[-0.5ex]
\hspace*{5mm}\verb+surf = ncread(filename,'zs');+
\\[-0.5ex]
\hspace*{5mm}\verb+% Display surface elevation+
\\[-0.5ex]
\hspace*{5mm}\verb+%    (transposition needed because MATLAB is column-oriented)+
\\[-0.5ex]
\hspace*{5mm}\verb+imagesc(x*1e-3,y*1e-3,surf'); axis xy equal; caxis([0 3200]); colorbar+

% \vfill

% \clearpage

\section{Some useful tools}
\label{sect_tools}

The directory sicopolis/tools contains some useful tools.

\subsection{Program make\_ismip\_output}
\label{ssect_make_ismip_output}

Generating ISMIP output (see http://tinyurl.com/clic-ismip6) from the NetCDF time-slice files produced by SICOPOLIS (see Sect.~\ref{sect_output}). For simulation \emph{run\_name}, to be executed by\\
\hspace*{10mm}\verb+./tools.sh -p make_ismip_output -m run_name+\\
For further options, try \verb+./tools.sh -h+.

\subsection{Program resolution\_doubler}
\label{ssect_resolution_doubler}

Doubling the horizontal resolution of a NetCDF time-slice output file produced by SICOPOLIS (see Sect.~\ref{sect_output}). For simulation \emph{run\_name}, to be executed by\\
\hspace*{10mm}\verb+./tools.sh -p resolution_doubler -m run_name+\\
For further options, try \verb+./tools.sh -h+.

For example, run v5\_grl10\_b2\_paleo21 (10\,km resolution) requires the resolution-doub\-led output of run v5\_grl20\_b2\_paleo21 (20\,km resolution) for $t=-9\,\mathrm{ka}$ as initial condition. In order to create it, execute the resolution doubler for run v5\_grl20\_\-b2\_paleo21 (i.e., with the option \verb+-m v5_grl20_b2_paleo21+) and enter
\\
\hspace*{10mm}Number of time-slice file (with leading zeros, 4 digits) $>$
\verb+0004+
\\
This will convert the original time-slice file v5\_grl20\_b2\_paleo210004.nc to the resolution-doubled file v5\_grl20\_b2\_paleo21\_dbl\_0004.nc that serves as initial conditions for run 
\linebreak{}
v5\_grl10\_b2\_paleo21.

\clearpage

\begin{appendix}

\section{Installation of NetCDF}
\label{sect_install_nc}

NetCDF (Network Common Data Form) is a common format for scientific data
\linebreak{}
(https://www.unidata.ucar.edu/software/netcdf/) that is also used by SICOPOLIS. The NetCDF C and Fortran libraries are required.

For \textbf{GCC}, installation from a package manager is recommended. Under openSUSE Leap 15.2, install netcdf, netcdf-devel, netcdf-devel-static, netcdf-fortran, netcdf-fortran-devel, netcdf-fortran-static, ncview. This requires the repositories ``Software for Scientists and Engineers'' and ``sebschub's Home Project''. Details (especially the required repositories) will differ for other systems.

For the \textbf{Intel compiler}, manual installation is required. The C and Fortran libraries are available for download on the NetCDF website as zip or tar archives. Unzip them into temporary source directories. 

\begin{itemize}

\item Prior to version 4.2, a single archive contained both the C and Fortran libraries. A minimal installation for version 4.1.3 (without NetCDF-4 support) can be done by changing to the source directory, then:

\begin{verbatim}
  export NCDIR=/opt/netcdf
  export CC=icc
  export FC=ifort
  export CFLAGS="-O2"
  export CPPFLAGS=
  export FCFLAGS="-O2"
  export FFLAGS=${FCFLAGS}
  ./configure --prefix=${NCDIR} --libdir=${NCDIR}/lib \
              --disable-netcdf-4
  make install
\end{verbatim}

\item Since version 4.2, the C and Fortran libraries must be installed separately. If Net\-CDF-4 support is dispensable, the following installation should work (tested under openSUSE Leap 15.2 and icc/ifort 19.1 with versions netcdf-c-4.8.0 and netcdf-fortran-4.5.3 as of January 25, 2021).

Step 1: Change to the source directory of the C library, then:

\begin{verbatim}
  export NCDIR=/opt/netcdf
  export CC=icc
  export FC=ifort
  export CFLAGS="-O2"
  export CPPFLAGS=
  export FCFLAGS="-O2"
  export FFLAGS=${FCFLAGS}
  ./configure --prefix=${NCDIR} --libdir=${NCDIR}/lib \
              --disable-netcdf-4 --enable-logging
  make install
\end{verbatim}

Step 2: Change to the source directory of the Fortran library, then:

\begin{verbatim}
  export NFDIR=/opt/netcdf
  export LD_LIBRARY_PATH=${NCDIR}/lib:${LD_LIBRARY_PATH}
  export CPPFLAGS=-I${NCDIR}/include
  export LDFLAGS=-L${NCDIR}/lib
  ./configure --prefix=${NFDIR} --libdir=${NFDIR}/lib \
              --disable-netcdf-4 --enable-logging
  make install
\end{verbatim}

\item For a complete build with NetCDF-4 support, additional libraries are required. See the NetCDF website for further instructions.

\end{itemize}

Installation under /opt usually requires admin rights. The same holds for the common alternative /usr/local. For a local installation, replace it by `/home/$<$my\_user\_name$>$/local'.

\section{Installation of Lis}
\label{sect_install_lis}

Lis (Library of Iterative Solvers for linear systems) is a software library for solving discretized linear equations \citep{nishida_2010}.

Download the latest version of Lis as a zip archive from https://www.ssisc.org/lis/ (as of January 23, 2021: lis-2.0.30.zip). Unzip the archive into a temporary directory.

For \textbf{GCC}, install lis by executing:

\begin{verbatim}
  export LISDIR=/opt/lis
  ./configure --prefix=${LISDIR} --libdir=${LISDIR}/lib \
              --enable-fortran --enable-f90 \
              --enable-omp --enable-saamg --enable-fma \
              CC=gcc FC=gfortran F77=gfortran \
              CFLAGS="-mcmodel=medium" CPPFLAGS="-mcmodel=medium" \
              FCFLAGS="-mcmodel=medium" FFLAGS="-mcmodel=medium"
  make install
\end{verbatim}

This has been tested under openSUSE Leap 15.2 and Linux Mint 20.1 (some modifications might be needed under different systems).

For the \textbf{Intel compiler}, replace `gcc' and `gfortran' by `icc' and `ifort', respectively.

Installation under /opt usually requires admin rights. The same holds for the common alternative /usr/local. For a local installation, replace it by `/home/$<$my\_user\_name$>$/local'.

\section{Domain XYZ}
\label{sect_xyz}

This framework allows creating new domains (Laurentide ice sheet, simple testing geometry etc.). The directory sicopolis/src/subroutines/xyz, which hosts the domain-specific subroutines, is by default empty. If you want to create a new domain, copy the subroutines from the most similar existing domain (northern hemisphere, EISMINT etc.), e.g.:
\\
\hspace*{10mm}\verb+cp sicopolis/src/subroutines/nhem/*.F90 \+
\\[-0.5ex]
\hspace*{14.3mm}\verb+ sicopolis/src/subroutines/xyz/ +
\\
Then modify the routines according to your needs. Input files (topography etc.) must be placed in sicopolis/sico\_in/xyz and specified in the run-specification header file *.h as usual. The domain must be defined by the domain code `\#define XYZ' in the header file. If the new domain requires new global variables, they can be defined in the module sicopolis/src/subroutines/xyz/sico\_vars.F90.

The subroutines for ISMIP HEINO are available in sicopolis/src/subroutines/xyz/heino, and the input files are in sicopolis/sico\_in/xyz. If you copy the subroutines from sicopo\-lis/src/subroutines/xyz/heino to sicopolis/src/subroutines/xyz, you can run ISMIP HEINO experiments (e.g., the run v5\_heino50\_st for which a header file is available).

% \vfill

\clearpage

\rule{0mm}{0mm}\vfill

\section{Table: Simulations and computing times}
\label{sect_table}

\begin{table}[ht]
\centering
\begin{tabular}{lrrr@{.}l} \hline
Run & Model time & Time step$^\dagger$ 
                 & \multicolumn{2}{c}{CPU time$^\ddagger$} \\ \hline
v5\_vialov3d25                       &  $100\;\mathrm{ka}$\hspace*{1.0em}
                                     &   $20\;\mathrm{ a}$\hspace*{1.0em}
                                     &  1&$0\;\mathrm{min}$ \\
v5\_emtp2sge25\_expA                 &  $200\;\mathrm{ka}$\hspace*{1.0em}
                                     &   $20\;\mathrm{ a}$\hspace*{1.0em}
                                     &  3&$8\;\mathrm{min}$ \\
v5\_grl20\_ss25ka                    &   $25\;\mathrm{ka}$\hspace*{1.0em}
                                     &    $5\;\mathrm{ a}$\hspace*{1.0em}
                                     &  6&$1\;\mathrm{min}$ \\
v5\_ant40\_ss25ka                    &   $25\;\mathrm{ka}$\hspace*{1.0em}
                                     &   $10\;\mathrm{ a}$\hspace*{1.0em}
                                     &  5&$0\;\mathrm{min}$ \\ \hline
v5\_grl20\_b2\_paleo21               &  $140\;\mathrm{ka}$\hspace*{1.0em}
                                     &    $5\;\mathrm{ a}$\hspace*{1.0em}
                                     &  0&$8\;\mathrm{hrs}$ \\
v5\_grl10\_b2\_paleo21$^\ast$        &    $9\;\mathrm{ka}$\hspace*{1.0em}
                                     &    $1\;\mathrm{ a}$\hspace*{1.0em}
                                     &  0&$9\;\mathrm{hrs}$ \\
v5\_grl10\_b2\_future21\_ctrl        &  $100\;\mathrm{ a}$\hspace*{1.47em}
                                     &    $1\;\mathrm{ a}$\hspace*{1.0em}
                                     &  0&$9\;\mathrm{min}$ \\
v5\_grl10\_b2\_future21\_asmb        &  $100\;\mathrm{ a}$\hspace*{1.47em}
                                     &    $1\;\mathrm{ a}$\hspace*{1.0em}
                                     &  0&$9\;\mathrm{min}$ \\ \hline
v5\_ant64\_b2\_spinup09\_init100a    &  $100\;\mathrm{ a}$\hspace*{1.47em}
                                     & $2\,/\,10\;\mathrm{ a}^{\dagger}$\hspace*{0.6em}
                                     &  3&$7\;\mathrm{sec}$ \\
v5\_ant64\_b2\_spinup09\_fixtopo     &  $140\;\mathrm{ka}$\hspace*{1.0em}
                                     &  $5\,/\,10\;\mathrm{ a}^{\dagger}$\hspace*{0.6em}
                                     &  0&$6\;\mathrm{hrs}$ \\
v5\_ant64\_b2\_spinup09              &  $500\;\mathrm{ a}$\hspace*{1.47em}
                                     & $2\,/\,10\;\mathrm{ a}^{\dagger}$\hspace*{0.6em}
                                     &  0&$3\;\mathrm{min}$ \\
v5\_ant64\_b2\_future09\_ctrl        &  $100\;\mathrm{ a}$\hspace*{1.47em}
                                     & $2\,/\,10\;\mathrm{ a}^{\dagger}$\hspace*{0.6em}
                                     &  4&$6\;\mathrm{sec}$ \\ \hline
\end{tabular}
\caption{Model times, time steps and computing (CPU) times for the EISMINT, Greenland and Antarctica simulations contained in the script multi\_sico\_1.sh, run with SICOPOLIS V5-dev (revision develop\_45\_rv5.1-44-ga6ca7e9) and the Intel Fortran Compiler 19.1 for Linux (optimization options -xHOST -O3 -no-prec-div) on a 12-Core Intel Xeon Gold 6256 (3.6~GHz) PC under openSUSE Leap 15.2.
\protect\\
{}$^\dagger$: If one value is given, this is the common dynamic (velocity, ice thickness) and thermodynamic (temperature, water content, age) time step. If two values are given (marked by the dagger ($^\dagger$) symbol), the first one is the dynamic, the second one the thermodynamic time step.
\protect\\
{}$^\ddagger$: All runs were done on a single core only. The v5\_ant64\_b2\_xxx runs that include ice shelves can be done on multiple cores using OpenMP for the SSA solver. However, at the employed, low resolution of 64\,km the solver does not scale well, and the gain in wall clock time by using multiple cores is very small.
\protect\\
{}$^\ast$: For this run, see the remark in Sect.~\ref{ssect_resolution_doubler} on the tool resolution\_doubler.}
\label{table_comp_times}
\end{table}

\vfill\rule{0mm}{0mm}

\clearpage

\end{appendix}

% \section*{References}

\bibliographystyle{ralf}
\bibliography{journals_full_rg,Refs_for_Quick_Start}

\end{document}
